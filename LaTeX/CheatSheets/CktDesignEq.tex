%CktDesignEq.tex
%
\documentclass[journal]{IEEEtran}%
%\documentclass[draftclsnofoot,peerreview,onecolumn]{IEEEtran}%
\title{Circuit Design Equations}%
\author{MA~Laforge}%
\def\AMSFooterTag{MA Laforge: Circuit Design Equations}
%
\usepackage{graphicx}\graphicspath{{./figures/}}%
%\usepackage{IEEEtrantools}%
\usepackage{amsmath}%
%\usepackage{latexsym}%Provides the \Box command (for sheet resistance)
\usepackage[reim_shrthnd,reim_curly]{../Common/CustomCommands}%
\usepackage[sort,compress,noadjust]{cite}%Does not work for some reason
\usepackage{../Common/Hyphenation}%correct bad hyphenation here
%
%*****More custom commands*****
\newcommand{\dt}{\mathrm{d}t}%
%
%*****End of custom commands*****
%
\begin{document}%
\maketitle%
%
%------------------------------------------------------------------------------
\section{Sinusoidal signals}
%
\begin{equation}
	V(t)=V_{pk}\sin(\omega{t}+\phi)
	\qquad
	I(t)=I_{pk}\sin(\omega{t}+\phi)
\label{eq:SinusoidalVI}%##################################################
\end{equation}
%
%------------------------------------------------------------------------------
\section{Signal Voltage, Current, \& Power}
%
\begin{equation}
	P=\int\limits_{T}^{}{{V^2(t)}\over{RT}}\dt
	\qquad
	P=\int\limits_{T}^{}{{RI^2(t)}\over{T}}\dt
\label{eq:PowerVI}%##################################################
\end{equation}
%
Sub \eqref{eq:SinusoidalVI} in \eqref{eq:PowerVI} \& solving:
\begin{equation}
	P={V_{pk}^2\over{2R}}
	\qquad
	P={RI_{pk}^2\over{2}}
	\nonumber
\end{equation}
%
%------------------------------------------------------------------------------
RMS Voltage \& Current,
%
\begin{equation}
	V_{RMS}\defAs\sqrt{RP}={V_{pk}\over\sqrt{2}}
	\qquad
	I_{RMS}\defAs\sqrt{P\over{R}}={I_{pk}\over\sqrt{2}}
	\nonumber
\end{equation}
%
%------------------------------------------------------------------------------
\section{Decibels \& Nepers}%
%\section{$\mathrm{dB, dbW, dBm}$}
%
\noindent Power (Decibels),%
\begin{equation}
	P_\mathrm{dB}\defAs{10}\log_{10}{P_1\over{P_2}}\,\mathrm{[dB]}
	\quad\Leftrightarrow\quad
	{P_1\over{P_2}}=10^{P_\mathrm{dB}\over{10}}
	\nonumber
\end{equation}
%
\begin{equation}
	{}=10\log_{10}{{{V^2_1}R_2}\over{{V^2_2}R_1}}\,\mathrm{[dB]}
	\quad\Leftrightarrow\quad
	{V_1\over{V_2}}\sqrt{R_2\over{R_1}}=10^{P_\mathrm{dB}\over{20}}
	\nonumber
\end{equation}
%
\begin{equation}
	{}=10\log_{10}{{{I^2_1}R_1}\over{{I^2_2}R_2}}\,\mathrm{[dB]}
	\quad\Leftrightarrow\quad
	{I_1\over{I_2}}\sqrt{R_1\over{R_2}}=10^{P_\mathrm{dB}\over{20}}
	\nonumber
\end{equation}
%
Power (Nepers),%
\begin{equation}
	P_\mathrm{Np}\defAs{1\over{2}}\ln{P_1\over{P_2}}\,\mathrm{[Np]}
	\quad\Leftrightarrow\quad
	{P_1\over{P_2}}=e^{2P_\mathrm{Np}}
	\nonumber
\end{equation}
%
\begin{equation}
	{}=\ln{{V_1\sqrt{R_2}\over{V_2\sqrt{R_1}}}}\,\mathrm{[Np]}
	\quad\Leftrightarrow\quad
	{V_1\over{V_2}}\sqrt{R_2\over{R_1}}=e^{P_\mathrm{Np}}
	\nonumber
\end{equation}
%
\begin{equation}
	{}=\ln{{I_1\sqrt{R_1}\over{I_2\sqrt{R_2}}}}\,\mathrm{[Np]}
	\quad\Leftrightarrow\quad
	{I_1\over{I_2}}\sqrt{R_1\over{R_2}}=e^{P_\mathrm{Np}}
	\nonumber
\end{equation}
%
Decibels $\Leftrightarrow$ Nepers,%
%
\begin{equation}
	e^{2P_\mathrm{Np}}={P_1\over{P_2}}=10^{P_\mathrm{dB}\over{10}}
	\nonumber
\end{equation}
%
\begin{equation}
	P_\mathrm{dB}={20\over\ln{10}}P_\mathrm{Np}=(20\log_{10}e)P_\mathrm{Np}
	\nonumber
\end{equation}
%
\begin{equation}
	1\,\mathrm{[Np]}\approx{8.686}\,\mathrm{[dB]}
	\nonumber
\end{equation}
%
\end{document}%
